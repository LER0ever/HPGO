\documentclass[12pt,letterpaper]{article}
\usepackage{fullpage}
\usepackage[top=2cm, bottom=4.5cm, left=2.5cm, right=2.5cm]{geometry}
\usepackage{amsmath,amsthm,amsfonts,amssymb,amscd}
\usepackage{lastpage}
\usepackage{enumerate}
\usepackage{fancyhdr}
\usepackage{mathrsfs}
\usepackage{xcolor}
\usepackage{graphicx}
\usepackage{listings}
\usepackage{hyperref}
\usepackage[yyyymmdd]{datetime}

\hypersetup{%
  colorlinks=true,
  linkcolor=blue,
  linkbordercolor={0 0 1}
}
 
\renewcommand\lstlistingname{Algorithm}
\renewcommand\lstlistlistingname{Algorithms}
\def\lstlistingautorefname{Alg.}

\lstdefinestyle{Python}{
    language        = Python,
    frame           = lines, 
    basicstyle      = \footnotesize,
    keywordstyle    = \color{blue},
    stringstyle     = \color{green},
    commentstyle    = \color{red}\ttfamily
}

\setlength{\parindent}{0.0in}
\setlength{\parskip}{0.05in}

\newcommand\doctitle{Hybrid Parallelism Global Orchestration}
\newcommand\ID{\href{https://rongyi.io}{rongyi.io}}


\pagestyle{fancyplain}
\headheight 35pt
\lhead{\ID}
\chead{\textbf{\doctitle}}
\rhead{\today}
\lfoot{DRAFT}
\cfoot{}
\rfoot{\small\thepage}
\headsep 1.5em

\begin{document}

\section {Notations}
\subsection {Profiling}
\begin{itemize}
	\item $L$: the total number of layers
	\item $l$: the order of layers from $1$ to $L$
	\item $T_l$: "the total computation time across the forward and backward passes for layer l on the target GPU"
	\item $a_l$: "the size of the output activations of layer $l$ (and the size of input gradients in the backward pass) in bytes"
	\item $w_l$: "the size of weight parameters for layer $l$ in bytes"
\end{itemize}
\subsection {Hyper Parameters}
\begin{itemize}
	\item $BS$: the Global Batch Size for training
\end{itemize}
\subsection {Environment}
\begin{itemize}
	\item $M$: number of Workers (GPUs) in total
	\item 
\end{itemize}

\section* {Algorithm}

\section* {Proof of Correctness}

\section* {Complexity Analysis}

\section*{Problem 2}

Answer to the problem goes here.

\begin{enumerate}
  \item
   Problem 1 part 1 answer here.
  \item
    Problem 1 part 2 answer here.

    Here is an example typesetting mathematics in \LaTeX
\begin{equation*}
    X(m,n) = \left\{\begin{array}{lr}
        x(n), & \text{for } 0\leq n\leq 1\\
        \frac{x(n-1)}{2}, & \text{for } 0\leq n\leq 1\\
        \log_2 \left\lceil n \right\rceil \qquad & \text{for } 0\leq n\leq 1
        \end{array}\right\} = xy
\end{equation*}

    \item Problem 1 part 3 answer here.

    Here is an example of how you can typeset algorithms.
    There are many packages to do this in \LaTeX.
     
    \lstset{caption={Caption for code}}
    \lstset{label={lst:alg1}}
     \begin{lstlisting}[style = Python]
    from package import Class # Mesh required for..
    
    cinstance = Class.from_obj('class.obj')
    cinstance.go()
    \end{lstlisting}
     
  \item Problem 1 part 4 answer here.

    Here is an example of how you can insert a figure.
    \begin{figure}[!h]
    \centering
%    \includegraphics[width=0.3\linewidth]{heidi.jpg}
    \caption{Heidi attacked by a string.}
    \end{figure}
\end{enumerate}


\section*{Problem 3}
% Rest of the work...

\end{document}